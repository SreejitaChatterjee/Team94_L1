\documentclass[10pt,a4paper,twocolumn]{article}

% Packages
\usepackage[utf8]{inputenc}
\usepackage[T1]{fontenc}
\usepackage{geometry}
\usepackage{graphicx}
\usepackage[hidelinks]{hyperref}
\usepackage{listings}
\usepackage{xcolor}
\usepackage{booktabs}
\usepackage{amsmath}
\usepackage{enumitem}
\usepackage{fancyhdr}
\usepackage{titlesec}
\usepackage{helvet}
\usepackage{setspace}
\renewcommand{\familydefault}{\sfdefault}
\setstretch{1.1}

% Page geometry - tighter margins
\geometry{margin=0.6in, top=0.8in, bottom=0.8in}

% Colors
\definecolor{codegreen}{rgb}{0,0.6,0}
\definecolor{codegray}{rgb}{0.5,0.5,0.5}
\definecolor{codepurple}{rgb}{0.58,0,0.82}
\definecolor{backcolour}{rgb}{0.95,0.95,0.92}

% Code listing style
\lstdefinestyle{mystyle}{
    backgroundcolor=\color{backcolour},
    commentstyle=\color{codegreen},
    keywordstyle=\color{blue},
    numberstyle=\tiny\color{codegray},
    stringstyle=\color{codepurple},
    basicstyle=\ttfamily\tiny,
    breakatwhitespace=false,
    breaklines=true,
    captionpos=b,
    keepspaces=true,
    numbers=none,
    showspaces=false,
    showstringspaces=false,
    showtabs=false,
    tabsize=2
}
\lstset{style=mystyle}

% Header/Footer
\pagestyle{fancy}
\fancyhf{}
\fancyhead[L]{\small Inter IIT Tech Meet 14.0 -- Qtrino Labs}
\fancyhead[R]{\small Team 94}
\fancyfoot[C]{\thepage}

% Title spacing - more compact
\titlespacing*{\section}{0pt}{1.2ex plus 0.3ex minus .2ex}{0.8ex plus .2ex}
\titlespacing*{\subsection}{0pt}{0.8ex plus 0.2ex minus .2ex}{0.4ex plus .2ex}

% Title
\title{\vspace{-1.2cm}\textbf{PQC-DTLS 1.3 Implementation on RISC-V Bare-Metal}\\[0.2cm]
\large Inter IIT Tech Meet 14.0 -- Qtrino Labs Challenge}
\author{\textbf{Team 94}}
\date{}

\begin{document}

\maketitle
\thispagestyle{fancy}
\vspace{-0.5cm}

%==============================================================================
\section{Problem Understanding}
%==============================================================================

The challenge requires implementing a Post-Quantum Cryptography (PQC) enabled DTLS 1.3 client on a resource-constrained RISC-V bare-metal environment, establishing secure communication with a host DTLS server using ML-KEM (Kyber) for quantum-resistant key exchange.

\textbf{Key Objectives:} (1) DTLS 1.3 client on LiteX VexRiscv SoC, (2) ML-KEM-512 post-quantum key exchange, (3) wolfSSL/wolfCrypt integration, (4) LiteETH networking, (5) Embedded optimization.

%==============================================================================
\section{Architecture and Design}
%==============================================================================

The system comprises two components via TAP virtual network:

\textbf{Host (Linux):} PQC-DTLS 1.3 server with wolfSSL at \texttt{192.168.1.100:11111}, performing ML-KEM encapsulation and AES-GCM encryption.

\textbf{LiteX Simulation:} Bare-metal DTLS client on VexRiscv (32-bit RV32IM, $\sim$100MHz) with LiteETH at \texttt{192.168.1.50:22222}.

\textbf{Software Stack:} Application (main.c) $\rightarrow$ wolfSSL with custom I/O $\rightarrow$ wolfCrypt (ML-KEM, AES-GCM, SHA-256) $\rightarrow$ LiteETH UDP $\rightarrow$ LiteX CSR HAL.

%==============================================================================
\section{PQC Algorithm Selection}
%==============================================================================

\textbf{ML-KEM-512} (formerly Kyber-512) selected as KEM for: NIST FIPS 203 standardization, memory efficiency (800B pubkey, 768B ciphertext), Level 1 security (128-bit classical), and native wolfSSL support.

\textbf{Symmetric:} AES-128-GCM for authenticated encryption, SHA-256 for key derivation, SHA3/SHAKE for ML-KEM internals.

%==============================================================================
\section{Firmware Implementation}
%==============================================================================

\subsection{Boot Sequence}
IRQ/UART init $\rightarrow$ LiteETH PHY init $\rightarrow$ UDP stack with MAC/IP $\rightarrow$ ARP resolution $\rightarrow$ wolfSSL init $\rightarrow$ DTLS 1.3 context $\rightarrow$ Handshake.

\subsection{Custom I/O Callbacks}
wolfSSL's socket I/O replaced with LiteETH callbacks: \textbf{Send} copies to TX buffer via \texttt{udp\_send()}; \textbf{Receive} polls ring buffer (8 slots) from \texttt{udp\_rx\_callback} ISR with timeout.

\subsection{Memory Layout}
\begin{table}[h]
\centering
\scriptsize
\begin{tabular}{@{}lll@{}}
\toprule
\textbf{Region} & \textbf{Address} & \textbf{Size} \\
\midrule
Main RAM & 0x40000000 & 100 MB \\
Stack & (top of RAM) & 500 KB \\
Heap & (after BSS) & 500 KB \\
\bottomrule
\end{tabular}
\end{table}

%==============================================================================
\section{wolfSSL Configuration}
%==============================================================================

Key \texttt{user\_settings.h} macros:
\begin{lstlisting}[language=C]
/* DTLS 1.3 + PQC */
#define WOLFSSL_DTLS13
#define WOLFSSL_HAVE_MLKEM
#define WOLFSSL_WC_MLKEM
/* Crypto */
#define HAVE_AESGCM
#define WOLFSSL_SHA3
/* Embedded */
#define WOLFSSL_SMALL_STACK
#define NO_FILESYSTEM
\end{lstlisting}

Enabled: DTLS 1.3 with fragmentation, ML-KEM, ECC (Curve25519), AES-GCM, SHA-256/512, SHA3/SHAKE, HKDF.

%==============================================================================
\section{Challenges and Solutions}
%==============================================================================

\begin{enumerate}[noitemsep,topsep=0pt,leftmargin=*]
    \item \textbf{Memory:} 500KB stack/heap, \texttt{WOLFSSL\_SMALL\_STACK}
    \item \textbf{No Sockets:} Custom wolfSSL I/O wrapping LiteETH UDP
    \item \textbf{No HW RNG:} PRNG with \texttt{CUSTOM\_RAND\_GENERATE\_SEED}
    \item \textbf{Packet Loss:} 8-slot ring buffer with timeout polling
    \item \textbf{Build:} Custom Makefile integrating wolfCrypt with LiteX
\end{enumerate}

%==============================================================================
\section{Security Analysis}
%==============================================================================

\begin{itemize}[noitemsep,topsep=0pt,leftmargin=*]
    \item \textbf{Quantum Resistance:} ML-KEM protects against Shor's algorithm
    \item \textbf{Forward Secrecy:} Ephemeral ML-KEM keys per session
    \item \textbf{AEAD:} AES-GCM provides confidentiality + integrity
    \item \textbf{Limitations:} Software PRNG (demo); production needs HW TRNG
\end{itemize}

%==============================================================================
\section{Performance}
%==============================================================================

\begin{table}[h]
\centering
\scriptsize
\begin{tabular}{@{}ll@{}}
\toprule
\textbf{Metric} & \textbf{Value} \\
\midrule
Firmware (.text) & 54 KB \\
Total Binary & 59 KB \\
ML-KEM KeyGen & $\sim$50 ms \\
ML-KEM Encaps/Decaps & $\sim$30-35 ms \\
\bottomrule
\end{tabular}
\end{table}

\textbf{Optimizations:} \texttt{WOLFSSL\_SP\_SMALL} for code size, \texttt{SP\_WORD\_SIZE=32} for RV32, \texttt{WOLFSSL\_AES\_SMALL\_TABLES}, \texttt{WOLFSSL\_SP\_NO\_MALLOC}.

%==============================================================================
\section{Conclusion}
%==============================================================================

We implemented a complete PQC-DTLS 1.3 client on bare-metal RISC-V using LiteX simulation. The system establishes quantum-resistant secure channels with a Linux DTLS server using ML-KEM-512 key exchange. Achievements: full DTLS 1.3 handshake, custom LiteETH I/O, ring buffer packet handling, and compact 59KB firmware.

\vspace{0.2cm}
\noindent\textbf{Repository:} \url{https://github.com/SreejitaChatterjee/Team94_L1}

%==============================================================================
\begin{thebibliography}{9}
\scriptsize
\bibitem{fips203} NIST, ``FIPS 203: ML-KEM Standard,'' 2024.
\bibitem{wolfssl} wolfSSL Inc., ``wolfSSL Embedded SSL/TLS,'' \url{https://www.wolfssl.com/}
\bibitem{litex} Enjoy-Digital, ``LiteX SoC Builder,'' \url{https://github.com/enjoy-digital/litex}
\bibitem{dtls13} E. Rescorla et al., ``DTLS 1.3,'' RFC 9147, 2022.
\end{thebibliography}

%==============================================================================
% Annexures
%==============================================================================
\newpage
\onecolumn
\appendix

\section*{Annexures}
\addcontentsline{toc}{section}{Annexures}

%------------------------------------------------------------------------------
\subsection*{Annexure A: Directory Structure}
%------------------------------------------------------------------------------

\begin{lstlisting}[language=bash,basicstyle=\ttfamily\small]
Team94_L1/
|-- LP_Constraint_Env_Sim/          # LiteX simulation environment
|   |-- boot/                       # RISC-V client firmware
|   |   |-- main.c                  # PQC-DTLS 1.3 client
|   |   |-- Makefile                # Build configuration
|   |   |-- server/                 # Linux DTLS server
|   |   |   `-- pqc_dtls_server.c
|   |   |-- wolfssl/                # wolfSSL headers
|   |   `-- wolfcrypt/src/          # wolfCrypt source files
|   |-- build/                      # LiteX build output
|   |-- report/                     # Technical report
|   `-- litex/, liteeth/, migen/    # LiteX framework
|
`-- README.md                       # Project documentation
\end{lstlisting}

%------------------------------------------------------------------------------
\subsection*{Annexure B: Build Instructions}
%------------------------------------------------------------------------------

\begin{lstlisting}[language=bash,basicstyle=\ttfamily\small]
# Build firmware
cd LP_Constraint_Env_Sim/boot && make clean && make

# Build server
cd boot/server && make dtls13

# Setup TAP interface
sudo ip tuntap add tap0 mode tap user $USER
sudo ip addr add 192.168.1.100/24 dev tap0
sudo ip link set tap0 up

# Run simulation
litex_sim --with-ethernet --ethernet-tap tap0 --ram-init=boot/boot.bin
\end{lstlisting}

%------------------------------------------------------------------------------
\subsection*{Annexure C: DTLS 1.3 Handshake Flow}
%------------------------------------------------------------------------------

\begin{lstlisting}[basicstyle=\ttfamily\small]
Client (LiteX)                         Server (Linux)
     |                                      |
     |-------- ClientHello + ML-KEM ------->|
     |<------- ServerHello + ML-KEM --------|
     |<------- EncryptedExtensions ---------|
     |<------- Finished --------------------|
     |-------- Finished ------------------->|
     |<======= Application Data ==========>|
     |        (AES-128-GCM encrypted)       |
\end{lstlisting}

%------------------------------------------------------------------------------
\subsection*{Annexure D: Network Configuration}
%------------------------------------------------------------------------------

\begin{table}[h]
\centering
\begin{tabular}{@{}lll@{}}
\toprule
\textbf{Parameter} & \textbf{Client (LiteX)} & \textbf{Server (Host)} \\
\midrule
IP Address & 192.168.1.50 & 192.168.1.100 \\
UDP Port & 22222 & 11111 \\
Interface & LiteETH & tap0 \\
\bottomrule
\end{tabular}
\end{table}

%------------------------------------------------------------------------------
\subsection*{Annexure E: Entropy Source (Demo)}
%------------------------------------------------------------------------------

\begin{lstlisting}[language=C,basicstyle=\ttfamily\small]
int CustomRngGenerateBlock(unsigned char *output, unsigned int sz) {
    static unsigned int seed = 0xDEADBEEF;
    for (unsigned int i = 0; i < sz; i++) {
        seed = seed * 1103515245 + 12345;
        output[i] = (unsigned char)(seed >> 16);
    }
    return 0;
}
// Production: Use hardware TRNG (ring oscillator-based)
\end{lstlisting}

\end{document}
