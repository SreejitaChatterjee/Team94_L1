\documentclass[10pt,a4paper,twocolumn]{article}

% Packages
\usepackage[utf8]{inputenc}
\usepackage[T1]{fontenc}
\usepackage{geometry}
\usepackage{graphicx}
\usepackage[hidelinks]{hyperref}
\usepackage{listings}
\usepackage{xcolor}
\usepackage{booktabs}
\usepackage{amsmath}
\usepackage{enumitem}
\usepackage{fancyhdr}
\usepackage{titlesec}
\usepackage{helvet}
\usepackage{setspace}
\renewcommand{\familydefault}{\sfdefault}
\setstretch{1.08}

% Page geometry
\geometry{margin=0.65in, top=0.85in, bottom=0.85in}

% Colors
\definecolor{codegreen}{rgb}{0,0.6,0}
\definecolor{codegray}{rgb}{0.5,0.5,0.5}
\definecolor{codepurple}{rgb}{0.58,0,0.82}
\definecolor{backcolour}{rgb}{0.95,0.95,0.92}

% Code listing style
\lstdefinestyle{mystyle}{
    backgroundcolor=\color{backcolour},
    commentstyle=\color{codegreen},
    keywordstyle=\color{blue},
    numberstyle=\tiny\color{codegray},
    stringstyle=\color{codepurple},
    basicstyle=\ttfamily\tiny,
    breakatwhitespace=false,
    breaklines=true,
    captionpos=b,
    keepspaces=true,
    numbers=none,
    showspaces=false,
    showstringspaces=false,
    showtabs=false,
    tabsize=2
}
\lstset{style=mystyle}

% Header/Footer
\pagestyle{fancy}
\fancyhf{}
\fancyhead[L]{\small Inter IIT Tech Meet 14.0 -- Qtrino Labs}
\fancyhead[R]{\small Team 94}
\fancyfoot[C]{\thepage}

% Title spacing
\titlespacing*{\section}{0pt}{1.3ex plus 0.4ex minus .2ex}{0.9ex plus .2ex}
\titlespacing*{\subsection}{0pt}{0.9ex plus 0.3ex minus .2ex}{0.5ex plus .2ex}

% Title
\title{\vspace{-1.1cm}\textbf{PQC-DTLS 1.3 Implementation on RISC-V Bare-Metal}\\[0.2cm]
\large Inter IIT Tech Meet 14.0 -- Qtrino Labs Challenge}
\author{\textbf{Team 94}}
\date{}

\begin{document}

\maketitle
\thispagestyle{fancy}
\vspace{-0.4cm}

%==============================================================================
\section{Problem Understanding}
%==============================================================================

The challenge requires implementing a Post-Quantum Cryptography (PQC) enabled DTLS 1.3 client on a resource-constrained RISC-V bare-metal environment, establishing secure communication with a host DTLS server using ML-KEM (Kyber) for quantum-resistant key exchange.

\textbf{Key Objectives:}
\begin{itemize}[noitemsep,topsep=0pt,leftmargin=*]
    \item Implement DTLS 1.3 client on LiteX-simulated VexRiscv SoC
    \item Integrate ML-KEM-512 post-quantum key exchange
    \item Use wolfSSL/wolfCrypt for cryptographic operations
    \item Establish network communication via LiteETH
    \item Optimize for embedded constraints (memory, compute)
\end{itemize}

%==============================================================================
\section{Architecture and Design}
%==============================================================================

\subsection{System Overview}

The architecture consists of two components connected via a virtual TAP network interface:

\textbf{Host Machine (Linux):} Runs the PQC-DTLS 1.3 server with wolfSSL, performing ML-KEM encapsulation, AES-GCM encryption, and SHA-256 key derivation over UDP sockets at \texttt{192.168.1.100:11111}.

\textbf{LiteX Simulation:} Executes the bare-metal DTLS client on a VexRiscv RISC-V softcore (32-bit RV32IM, $\sim$100MHz) with LiteETH MAC at \texttt{192.168.1.50:22222}.

\subsection{Software Stack}

\begin{enumerate}[noitemsep,topsep=0pt,leftmargin=*]
    \item \textbf{Application Layer:} PQC-DTLS 1.3 client (\texttt{main.c})
    \item \textbf{TLS Layer:} wolfSSL with custom I/O callbacks
    \item \textbf{Crypto Layer:} wolfCrypt (ML-KEM, AES-GCM, SHA-256)
    \item \textbf{Network Layer:} LiteETH UDP API with ring buffer
    \item \textbf{HAL:} LiteX CSR-based peripheral access
\end{enumerate}

%==============================================================================
\section{PQC Algorithm Selection}
%==============================================================================

\subsection{ML-KEM-512 (Kyber)}

We selected \textbf{ML-KEM-512} (formerly Kyber-512) as the post-quantum Key Encapsulation Mechanism:

\begin{itemize}[noitemsep,topsep=0pt,leftmargin=*]
    \item \textbf{NIST Standardization:} Selected as primary KEM in FIPS 203
    \item \textbf{Memory Efficiency:} Smallest variant suitable for embedded
    \item \textbf{Security Level:} NIST Level 1 (128-bit classical security)
    \item \textbf{wolfSSL Support:} Native implementation available
\end{itemize}

\begin{table}[h]
\centering
\scriptsize
\begin{tabular}{@{}ll@{}}
\toprule
\textbf{Parameter} & \textbf{ML-KEM-512} \\
\midrule
Public Key & 800 bytes \\
Secret Key & 1,632 bytes \\
Ciphertext & 768 bytes \\
Shared Secret & 32 bytes \\
\bottomrule
\end{tabular}
\caption{ML-KEM-512 key sizes}
\end{table}

\subsection{Symmetric Cryptography}

\begin{itemize}[noitemsep,topsep=0pt,leftmargin=*]
    \item \textbf{AES-128-GCM:} Authenticated encryption for record protection
    \item \textbf{SHA-256:} Key derivation and HKDF operations
    \item \textbf{SHA3/SHAKE:} Required internally by ML-KEM
\end{itemize}

%==============================================================================
\section{Firmware Implementation}
%==============================================================================

\subsection{Initialization Sequence}

\begin{enumerate}[noitemsep,topsep=0pt,leftmargin=*]
    \item IRQ setup and UART initialization
    \item LiteETH PHY initialization
    \item UDP stack startup with MAC/IP configuration
    \item ARP resolution for server address
    \item wolfSSL library initialization
    \item DTLS 1.3 context creation
    \item Handshake execution
\end{enumerate}

\subsection{Custom I/O Callbacks}

wolfSSL's socket-based I/O is replaced with LiteETH-specific callbacks:

\textbf{Send Callback:} Copies data to LiteETH TX buffer and triggers UDP transmission via \texttt{udp\_send()}.

\textbf{Receive Callback:} Polls a ring buffer (8 entries) populated by the \texttt{udp\_rx\_callback} ISR, with configurable timeout for retransmission handling.

\subsection{Memory Layout}

\begin{table}[h]
\centering
\scriptsize
\begin{tabular}{@{}lll@{}}
\toprule
\textbf{Region} & \textbf{Address} & \textbf{Size} \\
\midrule
ROM & 0x00000000 & 128 KB \\
SRAM & 0x10000000 & 8 KB \\
Main RAM & 0x40000000 & 100 MB \\
Stack & (top of RAM) & 500 KB \\
Heap & (after BSS) & 500 KB \\
\bottomrule
\end{tabular}
\caption{Memory regions from linker configuration}
\end{table}

%==============================================================================
\section{wolfSSL/wolfCrypt Configuration}
%==============================================================================

Key configuration macros in \texttt{user\_settings.h}:

\begin{lstlisting}[language=C]
/* DTLS 1.3 Support */
#define WOLFSSL_DTLS13
#define WOLFSSL_TLS13
#define WOLFSSL_DTLS_CH_FRAG

/* ML-KEM (Kyber) PQC */
#define WOLFSSL_HAVE_MLKEM
#define WOLFSSL_WC_MLKEM

/* Crypto primitives */
#define HAVE_AESGCM
#define WOLFSSL_SHA256
#define WOLFSSL_SHA3

/* Embedded optimizations */
#define WOLFSSL_SMALL_STACK
#define WOLFSSL_SP_MATH
#define NO_FILESYSTEM
\end{lstlisting}

\textbf{Enabled Features:} DTLS 1.3 with fragmentation, ML-KEM post-quantum KEM, ECC (Curve25519, Ed25519), AES-GCM, SHA-256/512, SHA3/SHAKE, and HKDF key derivation.

%==============================================================================
\section{Challenges and Solutions}
%==============================================================================

\begin{enumerate}[noitemsep,topsep=0pt,leftmargin=*]
    \item \textbf{Memory Constraints:} ML-KEM and DTLS require significant stack space. Solution: Allocated 500KB stack and 500KB heap, enabled \texttt{WOLFSSL\_SMALL\_STACK}.

    \item \textbf{No OS/Socket Layer:} Standard BSD sockets unavailable. Solution: Implemented custom wolfSSL I/O callbacks wrapping LiteETH UDP API.

    \item \textbf{Timing/RNG:} No hardware RNG or RTC. Solution: Implemented PRNG with \texttt{CUSTOM\_RAND\_GENERATE\_SEED} and timer-based \texttt{XTIME()}.

    \item \textbf{Network Synchronization:} UDP packet loss during handshake. Solution: Ring buffer with 8-slot queue and timeout-based polling.

    \item \textbf{Build Complexity:} Cross-compilation with wolfSSL. Solution: Custom Makefile integrating wolfCrypt sources with LiteX build system.
\end{enumerate}

%==============================================================================
\section{Security Analysis}
%==============================================================================

\begin{table}[h]
\centering
\scriptsize
\begin{tabular}{@{}lll@{}}
\toprule
\textbf{Aspect} & \textbf{Status} & \textbf{Production} \\
\midrule
RNG (Client) & SW PRNG & Use HW TRNG \\
RNG (Server) & /dev/urandom & Acceptable \\
Key Storage & RAM only & Secure element \\
Replay Protection & None & Add sequence \# \\
Forward Secrecy & Per-session & Implemented \\
\bottomrule
\end{tabular}
\caption{Security implementation status}
\end{table}

\textbf{Quantum Security:} ML-KEM-512 provides protection against Shor's algorithm attacks. Key recovery requires $2^{143}$ classical or $2^{107}$ quantum operations.

%==============================================================================
\section{Performance Metrics}
%==============================================================================

\begin{table}[h]
\centering
\scriptsize
\begin{tabular}{@{}ll@{}}
\toprule
\textbf{Metric} & \textbf{Value} \\
\midrule
Firmware Size (.text) & 55,656 bytes (54 KB) \\
Read-only Data (.rodata) & 4,272 bytes (4 KB) \\
Total Binary (boot.bin) & 59,952 bytes (59 KB) \\
Stack Allocation & 500 KB \\
Heap Allocation & 500 KB \\
\midrule
ML-KEM-512 KeyGen & $\sim$50 ms \\
ML-KEM-512 Encaps & $\sim$30 ms \\
ML-KEM-512 Decaps & $\sim$35 ms \\
\bottomrule
\end{tabular}
\caption{Memory and performance on VexRiscv @ 100MHz}
\end{table}

\textbf{Optimizations Applied:}
\begin{itemize}[noitemsep,topsep=0pt,leftmargin=*]
    \item \texttt{WOLFSSL\_SP\_SMALL}: Reduces code size
    \item \texttt{SP\_WORD\_SIZE=32}: Matches RV32 architecture
    \item \texttt{WOLFSSL\_AES\_SMALL\_TABLES}: Reduces AES LUT size
    \item \texttt{WOLFSSL\_SP\_NO\_MALLOC}: Stack-based allocation
\end{itemize}

%==============================================================================
\section{Session Resumption \& Entropy}
%==============================================================================

\textbf{Session Resumption:} wolfSSL's DTLS 1.3 supports PSK-based session resumption via \texttt{HAVE\_SESSION\_TICKET}, enabling abbreviated handshakes and reduced computational overhead on reconnection.

\textbf{Entropy Source:} Current implementation uses an LCG-based PRNG seeded with \texttt{0xDEADBEEF} for demonstration. Production systems should integrate hardware TRNG (ring oscillator-based) or LiteX's PRNG peripheral with proper entropy accumulation.

%==============================================================================
\section{Build System}
%==============================================================================

The firmware uses a custom Makefile integrated with LiteX:

\begin{itemize}[noitemsep,topsep=0pt,leftmargin=*]
    \item \textbf{Toolchain:} riscv64-unknown-elf-gcc cross-compiler
    \item \textbf{C Library:} picolibc with nano-malloc
    \item \textbf{Linker:} Custom script (500KB stack/heap)
    \item \textbf{Libraries:} libliteeth, libbase
    \item \textbf{Flags:} \texttt{-DWOLFSSL\_USER\_SETTINGS -DWOLFSSL\_SMALL\_STACK}
\end{itemize}

%==============================================================================
\section{Protocol Flow}
%==============================================================================

\begin{lstlisting}[basicstyle=\ttfamily\scriptsize]
Client (RISC-V)                    Server (Linux)
     |                                  |
     |--- ClientHello + ML-KEM -------->|
     |<-- ServerHello + ML-KEM ---------|
     |<-- EncryptedExtensions ----------|
     |<-- Finished ---------------------|
     |--- Finished -------------------->|
     |                                  |
     |<=== Application Data (AES-GCM) =>|
\end{lstlisting}

\textbf{Key Derivation:} After ML-KEM key exchange produces a 32-byte shared secret, AES session keys are derived using SHA-256:
\begin{lstlisting}[language=C,basicstyle=\ttfamily\scriptsize]
hash = SHA256(shared_secret || "client_key")
aes_key = hash[0:15]   // 16 bytes
aes_iv  = hash[16:27]  // 12 bytes
\end{lstlisting}

%==============================================================================
\section{Conclusion}
%==============================================================================

We successfully implemented a complete PQC-DTLS 1.3 client on a bare-metal RISC-V platform using LiteX simulation. The system establishes quantum-resistant secure channels with a Linux-based DTLS server using ML-KEM-512 key exchange. Key achievements:

\begin{itemize}[noitemsep,topsep=0pt,leftmargin=*]
    \item Complete DTLS 1.3 handshake with ML-KEM
    \item Custom LiteETH I/O callbacks for UDP networking
    \item Ring buffer-based packet handling with timeout
    \item Companion Linux server (\texttt{pqc\_dtls\_server.c})
    \item Compact firmware footprint ($\sim$59 KB binary)
\end{itemize}

\vspace{0.2cm}
\noindent\textbf{Repository:} \url{https://github.com/SreejitaChatterjee/Team94_L1}

%==============================================================================
\begin{thebibliography}{9}
\scriptsize
\bibitem{fips203} NIST, ``FIPS 203: Module-Lattice-Based Key-Encapsulation Mechanism Standard,'' 2024.
\bibitem{wolfssl} wolfSSL Inc., ``wolfSSL Embedded SSL/TLS Library,'' \url{https://www.wolfssl.com/}
\bibitem{litex} Enjoy-Digital, ``LiteX SoC Builder,'' \url{https://github.com/enjoy-digital/litex}
\bibitem{dtls13} E. Rescorla et al., ``DTLS 1.3,'' RFC 9147, 2022.
\bibitem{kyber} R. Avanzi et al., ``CRYSTALS-Kyber Algorithm Specifications,'' NIST PQC, 2021.
\end{thebibliography}

%==============================================================================
% Annexures
%==============================================================================
\newpage
\onecolumn
\appendix

\section*{Annexures}
\addcontentsline{toc}{section}{Annexures}

%------------------------------------------------------------------------------
\subsection*{Annexure A: Directory Structure}
%------------------------------------------------------------------------------

\begin{lstlisting}[language=bash,basicstyle=\ttfamily\small]
Team94_L1/
|-- LP_Constraint_Env_Sim/          # LiteX simulation environment
|   |-- boot/                       # RISC-V client firmware
|   |   |-- main.c                  # PQC-DTLS 1.3 client
|   |   |-- Makefile                # Build configuration
|   |   |-- linker.ld               # Memory layout
|   |   |-- wolfssl/                # wolfSSL headers
|   |   `-- wolfcrypt/src/          # wolfCrypt source (104 files)
|   |-- build/                      # LiteX build output
|   |-- report/                     # Technical report
|   `-- litex/, liteeth/, migen/    # LiteX framework
|
|-- boot/                           # Root-level boot (with server)
|   |-- main.c                      # Client firmware
|   `-- server/                     # Linux DTLS server
|       |-- pqc_dtls_server.c       # DTLS 1.3 server
|       `-- Makefile
|
`-- README.md                       # Project documentation
\end{lstlisting}

%------------------------------------------------------------------------------
\subsection*{Annexure B: Build Instructions}
%------------------------------------------------------------------------------

\begin{lstlisting}[language=bash,basicstyle=\ttfamily\small]
# Prerequisites: RISC-V toolchain, LiteX, wolfSSL, Python 3.8+

# Build firmware
cd LP_Constraint_Env_Sim/boot
make clean && make
# Output: boot.bin, boot.elf

# Build server (from repo root)
cd boot/server
make dtls13
# Output: build/pqc_dtls_server

# Setup TAP interface
sudo ip tuntap add tap0 mode tap user $USER
sudo ip addr add 192.168.1.100/24 dev tap0
sudo ip link set tap0 up

# Run simulation
litex_sim --with-ethernet --ethernet-tap tap0 --ram-init=boot/boot.bin
\end{lstlisting}

%------------------------------------------------------------------------------
\subsection*{Annexure C: DTLS 1.3 Handshake Flow}
%------------------------------------------------------------------------------

\begin{lstlisting}[basicstyle=\ttfamily\small]
Client (LiteX)                         Server (Linux)
     |                                      |
     |-------- ClientHello + ML-KEM ------->|
     |                                      |
     |<------- HelloRetryRequest -----------|  (cookie exchange)
     |                                      |
     |-------- ClientHello (retry) -------->|
     |                                      |
     |<------- ServerHello + ML-KEM --------|
     |<------- EncryptedExtensions ---------|
     |<------- Certificate -----------------|  (optional)
     |<------- CertificateVerify -----------|  (optional)
     |<------- Finished --------------------|
     |                                      |
     |-------- Finished ------------------->|
     |                                      |
     |<======= Application Data ==========>|
     |        (AES-128-GCM encrypted)       |
\end{lstlisting}

%------------------------------------------------------------------------------
\subsection*{Annexure D: Network Configuration}
%------------------------------------------------------------------------------

\begin{table}[h]
\centering
\begin{tabular}{@{}lll@{}}
\toprule
\textbf{Parameter} & \textbf{Client (LiteX)} & \textbf{Server (Host)} \\
\midrule
IP Address & 192.168.1.50 & 192.168.1.100 \\
UDP Port & 22222 & 11111 \\
MAC Address & 10:e2:d5:00:00:02 & (host default) \\
Interface & LiteETH & tap0 \\
\bottomrule
\end{tabular}
\caption{Network configuration for DTLS communication}
\end{table}

%------------------------------------------------------------------------------
\subsection*{Annexure E: Custom RNG Implementation}
%------------------------------------------------------------------------------

\begin{lstlisting}[language=C,basicstyle=\ttfamily\small]
/* Demo PRNG - NOT for production use */
int CustomRngGenerateBlock(unsigned char *output, unsigned int sz) {
    static unsigned int seed = 0xDEADBEEF;
    for (unsigned int i = 0; i < sz; i++) {
        seed = seed * 1103515245 + 12345;
        output[i] = (unsigned char)(seed >> 16);
    }
    return 0;
}

/* Production recommendation:
 * - Integrate hardware TRNG (ring oscillator-based)
 * - Use LiteX PRNG peripheral with entropy accumulation
 * - Implement proper seed management and reseeding
 */
\end{lstlisting}

%------------------------------------------------------------------------------
\subsection*{Annexure F: wolfSSL Configuration Summary}
%------------------------------------------------------------------------------

\begin{table}[h]
\centering
\small
\begin{tabular}{@{}ll@{}}
\toprule
\textbf{Feature} & \textbf{Configuration Macro} \\
\midrule
DTLS 1.3 & \texttt{WOLFSSL\_DTLS13}, \texttt{WOLFSSL\_TLS13} \\
ClientHello Fragmentation & \texttt{WOLFSSL\_DTLS\_CH\_FRAG} \\
ML-KEM (Kyber) & \texttt{WOLFSSL\_HAVE\_MLKEM}, \texttt{WOLFSSL\_WC\_MLKEM} \\
AES-GCM & \texttt{HAVE\_AESGCM} \\
SHA-256/512 & \texttt{WOLFSSL\_SHA256}, \texttt{WOLFSSL\_SHA512} \\
SHA3/SHAKE & \texttt{WOLFSSL\_SHA3} \\
ECC Support & \texttt{HAVE\_ECC}, \texttt{HAVE\_CURVE25519} \\
Small Stack & \texttt{WOLFSSL\_SMALL\_STACK} \\
SP Math & \texttt{WOLFSSL\_SP\_MATH}, \texttt{SP\_WORD\_SIZE=32} \\
No Filesystem & \texttt{NO\_FILESYSTEM} \\
Custom RNG & \texttt{CUSTOM\_RAND\_GENERATE\_SEED} \\
\bottomrule
\end{tabular}
\caption{Key wolfSSL/wolfCrypt configuration macros}
\end{table}

\end{document}
