\documentclass[10pt,a4paper,twocolumn]{article}

% Packages
\usepackage[utf8]{inputenc}
\usepackage[T1]{fontenc}
\usepackage{geometry}
\usepackage{graphicx}
\usepackage[hidelinks]{hyperref}
\usepackage{listings}
\usepackage{xcolor}
\usepackage{booktabs}
\usepackage{amsmath}
\usepackage{enumitem}
\usepackage{fancyhdr}
\usepackage{titlesec}
\usepackage{helvet}                    % Helvetica font
\usepackage{setspace}                  % Line spacing
\renewcommand{\familydefault}{\sfdefault}  % Use Helvetica as default
\setstretch{1.15}                      % 1.15 line spacing for entire document

% Page geometry
\geometry{margin=0.75in, top=1in, bottom=1in}

% Colors
\definecolor{codegreen}{rgb}{0,0.6,0}
\definecolor{codegray}{rgb}{0.5,0.5,0.5}
\definecolor{codepurple}{rgb}{0.58,0,0.82}
\definecolor{backcolour}{rgb}{0.95,0.95,0.92}

% Code listing style
\lstdefinestyle{mystyle}{
    backgroundcolor=\color{backcolour},
    commentstyle=\color{codegreen},
    keywordstyle=\color{blue},
    numberstyle=\tiny\color{codegray},
    stringstyle=\color{codepurple},
    basicstyle=\ttfamily\scriptsize,
    breakatwhitespace=false,
    breaklines=true,
    captionpos=b,
    keepspaces=true,
    numbers=left,
    numbersep=5pt,
    showspaces=false,
    showstringspaces=false,
    showtabs=false,
    tabsize=2
}
\lstset{style=mystyle}

% Header/Footer
\pagestyle{fancy}
\fancyhf{}
\fancyhead[L]{\small Inter IIT Tech Meet 14.0 -- Qtrino Labs}
\fancyhead[R]{\small Team 94}
\fancyfoot[C]{\thepage}

% Title spacing
\titlespacing*{\section}{0pt}{1.5ex plus 0.5ex minus .2ex}{1ex plus .2ex}
\titlespacing*{\subsection}{0pt}{1ex plus 0.3ex minus .2ex}{0.5ex plus .2ex}

% Title
\title{\vspace{-1cm}\textbf{PQC-DTLS 1.3 Implementation on RISC-V Bare-Metal Environment}\\[0.3cm]
\large Inter IIT Tech Meet 14.0 -- Qtrino Labs Challenge}
\author{
\textbf{Team 94}
}
\date{}

\begin{document}

\maketitle
\thispagestyle{fancy}

%==============================================================================
\section{Problem Understanding}
%==============================================================================

The challenge requires implementing a Post-Quantum Cryptography (PQC) enabled DTLS 1.3 client on a resource-constrained RISC-V bare-metal environment. The system must establish secure communication with a host-based DTLS server using quantum-resistant cryptographic algorithms, specifically ML-KEM (Kyber) for key encapsulation.

\textbf{Key Objectives:}
\begin{itemize}[noitemsep,topsep=0pt]
    \item Implement DTLS 1.3 client on LiteX-simulated VexRiscv SoC
    \item Integrate ML-KEM-512 post-quantum key exchange
    \item Use wolfSSL/wolfCrypt for cryptographic operations
    \item Establish network communication via LiteETH
    \item Optimize for embedded constraints (memory, compute)
\end{itemize}

%==============================================================================
\section{Architecture and Design}
%==============================================================================

\subsection{System Overview}

The architecture consists of two components connected via a virtual TAP network interface:

\textbf{Host Machine (Linux):} Runs the PQC-DTLS 1.3 server with wolfSSL, performing ML-KEM encapsulation, AES-GCM encryption, and SHA-256 key derivation over UDP sockets at \texttt{192.168.1.100:11111}.

\textbf{LiteX Simulation:} Executes the bare-metal DTLS client on a VexRiscv RISC-V softcore (32-bit RV32IM, $\sim$100MHz) with LiteETH MAC at \texttt{192.168.1.50:22222}.

\subsection{Software Stack}

\begin{enumerate}[noitemsep,topsep=0pt]
    \item \textbf{Application Layer:} PQC-DTLS 1.3 client (\texttt{main.c})
    \item \textbf{TLS Layer:} wolfSSL with custom I/O callbacks
    \item \textbf{Crypto Layer:} wolfCrypt (ML-KEM, AES-GCM, SHA-256)
    \item \textbf{Network Layer:} LiteETH UDP API with ring buffer
    \item \textbf{HAL:} LiteX CSR-based peripheral access
\end{enumerate}

%==============================================================================
\section{PQC Algorithm Choices}
%==============================================================================

\subsection{ML-KEM-512 (Kyber)}

We selected \textbf{ML-KEM-512} (formerly Kyber-512) as the post-quantum Key Encapsulation Mechanism for the following reasons:

\begin{itemize}[noitemsep,topsep=0pt]
    \item \textbf{NIST Standardization:} Selected as the primary KEM in FIPS 203
    \item \textbf{Memory Efficiency:} Smallest variant with 800-byte public keys and 768-byte ciphertexts
    \item \textbf{Security Level:} NIST Level 1 (128-bit classical security)
    \item \textbf{wolfSSL Support:} Native implementation available
\end{itemize}

\textbf{Key Sizes:}
\begin{itemize}[noitemsep,topsep=0pt]
    \item Public Key: 800 bytes
    \item Secret Key: 1,632 bytes
    \item Ciphertext: 768 bytes
    \item Shared Secret: 32 bytes
\end{itemize}

\subsection{Symmetric Cryptography}

\begin{itemize}[noitemsep,topsep=0pt]
    \item \textbf{AES-128-GCM:} Authenticated encryption for record protection
    \item \textbf{SHA-256:} Key derivation and HKDF operations
    \item \textbf{SHA3/SHAKE:} Required internally by ML-KEM
\end{itemize}

%==============================================================================
\section{Firmware Design}
%==============================================================================

\subsection{Initialization Sequence}

\begin{lstlisting}[language=C,caption={Boot sequence}]
1. IRQ setup and UART init
2. LiteETH PHY initialization
3. UDP stack startup with MAC/IP
4. ARP resolution for server
5. wolfSSL initialization
6. DTLS 1.3 context creation
7. Handshake execution
\end{lstlisting}

\subsection{Custom I/O Callbacks}

wolfSSL's socket-based I/O is replaced with LiteETH-specific callbacks:

\textbf{Send Callback:} Copies data to LiteETH TX buffer and triggers UDP transmission via \texttt{udp\_send()}.

\textbf{Receive Callback:} Polls a ring buffer (8 entries) populated by the \texttt{udp\_rx\_callback} ISR, with configurable timeout.

\subsection{Memory Layout}

\begin{table}[h]
\centering
\scriptsize
\begin{tabular}{@{}lll@{}}
\toprule
\textbf{Region} & \textbf{Address} & \textbf{Size} \\
\midrule
ROM & 0x00000000 & 128 KB \\
SRAM & 0x10000000 & 8 KB \\
Main RAM & 0x40000000 & 100 MB \\
ETH MAC & 0x80000000 & 8 KB \\
CSR & 0xF0000000 & 64 KB \\
\midrule
Stack & (top of RAM) & 500 KB \\
Heap & (after BSS) & 500 KB \\
\bottomrule
\end{tabular}
\caption{Memory regions from linker configuration}
\end{table}

%==============================================================================
\section{wolfSSL/wolfCrypt Integration}
%==============================================================================

\subsection{Configuration (\texttt{user\_settings.h})}

Key configuration macros for bare-metal PQC-DTLS operation:

\begin{lstlisting}[language=C,caption={wolfSSL/wolfCrypt configuration}]
/* DTLS 1.3 Support */
#define WOLFSSL_DTLS13
#define WOLFSSL_TLS13
#define WOLFSSL_DTLS_CH_FRAG
#define WOLFSSL_SEND_HRR_COOKIE

/* ML-KEM (Kyber) PQC */
#define WOLFSSL_HAVE_MLKEM
#define WOLFSSL_WC_MLKEM

/* ECC and RSA */
#define HAVE_ECC
#define HAVE_CURVE25519
#define HAVE_ED25519
#define WC_RSA_PSS

/* Crypto primitives */
#define HAVE_AESGCM
#define WOLFSSL_SHA256
#define WOLFSSL_SHA512
#define WOLFSSL_SHA3

/* Embedded optimizations */
#define WOLFSSL_SMALL_STACK
#define WOLFSSL_SP_MATH
#define NO_FILESYSTEM
\end{lstlisting}

\subsection{Enabled Features}

The configuration enables: DTLS 1.3 with fragmentation, ML-KEM post-quantum KEM, ECC (Curve25519, Ed25519), RSA with PSS, AES-GCM, SHA-256/512, SHA3/SHAKE, and HKDF key derivation.

%==============================================================================
\section{Challenges and Solutions}
%==============================================================================

\begin{enumerate}[noitemsep,topsep=0pt]
    \item \textbf{Memory Constraints:} ML-KEM and DTLS require significant stack space. Solution: Allocated 500KB stack and 500KB heap, enabled \texttt{WOLFSSL\_SMALL\_STACK}.

    \item \textbf{No OS/Socket Layer:} Standard BSD sockets unavailable. Solution: Implemented custom wolfSSL I/O callbacks wrapping LiteETH UDP API.

    \item \textbf{Timing/RNG:} No hardware RNG or RTC. Solution: Implemented PRNG with \texttt{CUSTOM\_RAND\_GENERATE\_SEED} and timer-based \texttt{XTIME()}.

    \item \textbf{Network Synchronization:} UDP packet loss during handshake. Solution: Ring buffer with 8-slot queue and timeout-based polling.

    \item \textbf{Build Complexity:} Cross-compilation with wolfSSL. Solution: Custom Makefile integrating wolfCrypt sources with LiteX build system.
\end{enumerate}

%==============================================================================
\section{Security Considerations}
%==============================================================================

\begin{itemize}[noitemsep,topsep=0pt]
    \item \textbf{Quantum Resistance:} ML-KEM provides protection against Shor's algorithm attacks on key exchange
    \item \textbf{Forward Secrecy:} Ephemeral ML-KEM keys per session
    \item \textbf{Authenticated Encryption:} AES-GCM provides confidentiality and integrity
    \item \textbf{PRNG Limitation:} Current PRNG is deterministic (demo only); production requires hardware TRNG
    \item \textbf{Certificate Verification:} Disabled for demo; production should enable with Dilithium signatures
\end{itemize}

%==============================================================================
\section{Performance Metrics}
%==============================================================================

\begin{table}[h]
\centering
\scriptsize
\begin{tabular}{@{}ll@{}}
\toprule
\textbf{Metric} & \textbf{Value} \\
\midrule
\multicolumn{2}{@{}l@{}}{\textit{Memory Layout (from linker map)}} \\
Firmware Size (.text) & 55,656 bytes (54 KB) \\
Read-only Data (.rodata) & 4,272 bytes (4 KB) \\
Initialized Data (.data) & 24 bytes \\
BSS (.bss) & 400 bytes \\
Total Binary (boot.bin) & 59,952 bytes (59 KB) \\
\midrule
\multicolumn{2}{@{}l@{}}{\textit{Stack/Heap Allocation}} \\
Stack Size (allocated) & 500 KB \\
Heap Size (allocated) & 500 KB \\
\midrule
\multicolumn{2}{@{}l@{}}{\textit{Crypto Operations (estimated)}} \\
ML-KEM-512 KeyGen & $\sim$50 ms \\
ML-KEM-512 Encaps & $\sim$30 ms \\
ML-KEM-512 Decaps & $\sim$35 ms \\
\bottomrule
\end{tabular}
\caption{Memory and performance on VexRiscv @ 100MHz (simulated)}
\end{table}

%==============================================================================
\section{Session Resumption}
%==============================================================================

wolfSSL's DTLS 1.3 implementation supports session resumption via PSK (Pre-Shared Key) mode. Our configuration enables:

\begin{itemize}[noitemsep,topsep=0pt]
    \item Session ticket storage for abbreviated handshakes
    \item 0-RTT data support (when enabled)
    \item Reduced computational overhead on reconnection
\end{itemize}

Full session resumption requires server-side ticket issuance and client-side caching, which can be enabled via \texttt{HAVE\_SESSION\_TICKET}.

%==============================================================================
\section{Optimizations for Low-Power RISC-V}
%==============================================================================

\begin{enumerate}[noitemsep,topsep=0pt]
    \item \textbf{Small Math Library:} \texttt{WOLFSSL\_SP\_SMALL} reduces code size with minor performance impact
    \item \textbf{32-bit Word Size:} \texttt{SP\_WORD\_SIZE=32} matches RV32 architecture
    \item \textbf{Small Tables:} \texttt{WOLFSSL\_AES\_SMALL\_TABLES} reduces AES lookup table size
    \item \textbf{Memory-Optimized ML-KEM:} \texttt{MLKEM\_SMALL\_MEM} variants reduce peak memory
    \item \textbf{No Dynamic Allocation in SP:} \texttt{WOLFSSL\_SP\_NO\_MALLOC} uses stack allocation
\end{enumerate}

%==============================================================================
\section{Entropy Source}
%==============================================================================

The current implementation uses a Linear Congruential Generator (LCG) seeded with a constant (\texttt{0xDEADBEEF}) for demonstration purposes. The custom RNG function:

\begin{lstlisting}[language=C]
int CustomRngGenerateBlock(
    unsigned char *output,
    unsigned int sz) {
    static unsigned int seed = 0xDEADBEEF;
    for (unsigned int i = 0; i < sz; i++) {
        seed = seed * 1103515245 + 12345;
        output[i] = (unsigned char)(seed >> 16);
    }
    return 0;
}
\end{lstlisting}

\textbf{Production Recommendation:} Integrate hardware TRNG (e.g., ring oscillator-based) or use LiteX's PRNG peripheral with proper entropy accumulation.

%==============================================================================
\section{Build System}
%==============================================================================

The firmware is built using a custom Makefile integrated with LiteX's build infrastructure:

\begin{itemize}[noitemsep,topsep=0pt]
    \item \textbf{Toolchain:} RISC-V GCC cross-compiler (riscv64-unknown-elf-gcc)
    \item \textbf{C Library:} picolibc with nano-malloc for embedded systems
    \item \textbf{Linker:} Custom linker script with 500KB stack and 500KB heap
    \item \textbf{Libraries:} libliteeth for Ethernet, libbase for UART/console
    \item \textbf{Server:} Companion Linux DTLS server using system wolfSSL
\end{itemize}

\textbf{Key compiler flags:}
\begin{lstlisting}[language=bash,basicstyle=\ttfamily\tiny]
-DWOLFSSL_USER_SETTINGS
-DWOLFSSL_SMALL_STACK
-DWOLFSSL_STATIC_MEMORY
\end{lstlisting}

%==============================================================================
\section{Conclusion}
%==============================================================================

We implemented a full PQC-DTLS 1.3 client on a bare-metal RISC-V platform using LiteX simulation. The system establishes a secure channel with a Linux-based DTLS server using wolfSSL's DTLS 1.3 implementation with ML-KEM-512 post-quantum key exchange. Key achievements include:

\begin{itemize}[noitemsep,topsep=0pt]
    \item Complete DTLS 1.3 handshake with ML-KEM key exchange
    \item Custom LiteETH I/O callbacks for UDP networking
    \item Ring buffer-based packet handling with timeout support
    \item Companion Linux server (\texttt{pqc\_dtls\_server.c})
    \item Compact firmware footprint ($\sim$59 KB binary)
\end{itemize}

\vspace{0.3cm}
\noindent\textbf{Repository:} \url{https://github.com/Team94/LP_Constraint_Env_Sim}

%==============================================================================
% Bibliography
%==============================================================================
\begin{thebibliography}{9}
\scriptsize

\bibitem{fips203}
NIST, ``FIPS 203: Module-Lattice-Based Key-Encapsulation Mechanism Standard (ML-KEM),'' 2024.

\bibitem{wolfssl}
wolfSSL Inc., ``wolfSSL Embedded SSL/TLS Library,'' \url{https://www.wolfssl.com/}

\bibitem{litex}
Enjoy-Digital, ``LiteX: A Migen/MiSoC based SoC builder,'' \url{https://github.com/enjoy-digital/litex}

\bibitem{vexriscv}
SpinalHDL, ``VexRiscv: A FPGA friendly 32-bit RISC-V CPU,'' \url{https://github.com/SpinalHDL/VexRiscv}

\bibitem{dtls13}
E. Rescorla, H. Tschofenig, N. Modadugu, ``The Datagram Transport Layer Security (DTLS) Protocol Version 1.3,'' RFC 9147, 2022.

\bibitem{kyber}
R. Avanzi et al., ``CRYSTALS-Kyber: Algorithm Specifications and Supporting Documentation,'' NIST PQC Round 3 Submission, 2021.

\end{thebibliography}

%==============================================================================
% Annexures
%==============================================================================
\newpage
\onecolumn
\appendix

\section*{Annexures}
\addcontentsline{toc}{section}{Annexures}

%------------------------------------------------------------------------------
\subsection*{Annexure A: Directory Structure}
%------------------------------------------------------------------------------

\begin{lstlisting}[language=bash,basicstyle=\ttfamily\small]
Team94_L1/
|-- LP_Constraint_Env_Sim/          # LiteX simulation environment
|   |-- boot/                       # RISC-V client firmware
|   |   |-- main.c                  # PQC-DTLS 1.3 client
|   |   |-- Makefile                # Build configuration
|   |   |-- linker.ld               # Memory layout
|   |   |-- boot.bin                # Compiled binary (output)
|   |   |-- src/                    # wolfSSL TLS source
|   |   |-- wolfssl/                # wolfSSL headers
|   |   |   `-- wolfcrypt/
|   |   |       `-- user_settings.h # wolfSSL configuration
|   |   `-- wolfcrypt/src/          # wolfCrypt source files
|   |-- build/                      # LiteX build output
|   |-- report/                     # Technical report
|   |-- litex/, liteeth/, migen/    # LiteX framework
|   `-- README.md
|
|-- boot/                           # Root-level boot directory
|   |-- main.c                      # Client firmware (copy)
|   `-- server/                     # Linux DTLS server
|       |-- pqc_dtls_server.c       # DTLS 1.3 server
|       |-- Makefile
|       `-- user_settings.h
|
`-- PQC_DTLS_README.md              # Protocol documentation
\end{lstlisting}

%------------------------------------------------------------------------------
\subsection*{Annexure B: Build Instructions}
%------------------------------------------------------------------------------

\textbf{Prerequisites:}
\begin{itemize}[noitemsep]
    \item RISC-V GCC toolchain (riscv64-unknown-elf-gcc)
    \item LiteX framework with VexRiscv support
    \item wolfSSL library (v5.6.0 or later with PQC support)
    \item Python 3.8+ with Migen/LiteX dependencies
\end{itemize}

\textbf{Building the Firmware:}
\begin{lstlisting}[language=bash,basicstyle=\ttfamily\small]
# Navigate to boot directory
cd LP_Constraint_Env_Sim/boot

# Build the firmware
make clean && make

# Output: boot.bin (load into LiteX simulation)
\end{lstlisting}

\textbf{Building the Server:}
\begin{lstlisting}[language=bash,basicstyle=\ttfamily\small]
# Navigate to server directory (root level)
cd Team94_L1/boot/server

# Build with wolfSSL
make

# Run the server
./build/pqc_dtls_server
\end{lstlisting}

\textbf{Running the Simulation:}
\begin{lstlisting}[language=bash,basicstyle=\ttfamily\small]
# Create TAP interface
sudo ip tuntap add tap0 mode tap
sudo ip addr add 192.168.1.100/24 dev tap0
sudo ip link set tap0 up

# Run LiteX simulation
litex_sim --with-ethernet --ethernet-tap tap0
\end{lstlisting}

%------------------------------------------------------------------------------
\subsection*{Annexure C: Network Configuration}
%------------------------------------------------------------------------------

\begin{table}[h]
\centering
\begin{tabular}{@{}lll@{}}
\toprule
\textbf{Parameter} & \textbf{Client (LiteX)} & \textbf{Server (Host)} \\
\midrule
IP Address & 192.168.1.50 & 192.168.1.100 \\
UDP Port & 22222 & 11111 \\
MAC Address & 0x10:0xe2:d5:00:00:02 & (host default) \\
Interface & LiteETH & tap0 \\
\bottomrule
\end{tabular}
\caption{Network configuration for DTLS communication}
\end{table}

%------------------------------------------------------------------------------
\subsection*{Annexure D: wolfSSL Configuration Summary}
%------------------------------------------------------------------------------

\begin{table}[h]
\centering
\small
\begin{tabular}{@{}ll@{}}
\toprule
\textbf{Feature} & \textbf{Configuration Macro} \\
\midrule
DTLS 1.3 & \texttt{WOLFSSL\_DTLS13}, \texttt{WOLFSSL\_TLS13} \\
ML-KEM (Kyber) & \texttt{WOLFSSL\_HAVE\_MLKEM}, \texttt{WOLFSSL\_WC\_MLKEM} \\
AES-GCM & \texttt{HAVE\_AESGCM} \\
SHA-256/512 & \texttt{WOLFSSL\_SHA256}, \texttt{WOLFSSL\_SHA512} \\
SHA3/SHAKE & \texttt{WOLFSSL\_SHA3} \\
ECC Support & \texttt{HAVE\_ECC}, \texttt{HAVE\_CURVE25519} \\
Small Stack & \texttt{WOLFSSL\_SMALL\_STACK} \\
No Filesystem & \texttt{NO\_FILESYSTEM} \\
Custom RNG & \texttt{CUSTOM\_RAND\_GENERATE\_SEED} \\
\bottomrule
\end{tabular}
\caption{Key wolfSSL/wolfCrypt configuration macros}
\end{table}

%------------------------------------------------------------------------------
\subsection*{Annexure E: DTLS 1.3 Handshake Flow}
%------------------------------------------------------------------------------

\begin{lstlisting}[basicstyle=\ttfamily\small]
Client (LiteX)                         Server (Linux)
     |                                      |
     |-------- ClientHello + ML-KEM ------->|
     |                                      |
     |<------- HelloRetryRequest -----------|
     |                                      |
     |-------- ClientHello (retry) -------->|
     |                                      |
     |<------- ServerHello + ML-KEM --------|
     |<------- EncryptedExtensions ---------|
     |<------- Certificate -----------------|
     |<------- CertificateVerify -----------|
     |<------- Finished --------------------|
     |                                      |
     |-------- Finished ------------------->|
     |                                      |
     |<======= Application Data ==========>|
     |        (AES-128-GCM encrypted)       |
\end{lstlisting}

\end{document}
